% Options for packages loaded elsewhere
\PassOptionsToPackage{unicode}{hyperref}
\PassOptionsToPackage{hyphens}{url}
%
\documentclass[
]{article}
\usepackage{amsmath,amssymb}
\usepackage{iftex}
\ifPDFTeX
  \usepackage[T1]{fontenc}
  \usepackage[utf8]{inputenc}
  \usepackage{textcomp} % provide euro and other symbols
\else % if luatex or xetex
  \usepackage{unicode-math} % this also loads fontspec
  \defaultfontfeatures{Scale=MatchLowercase}
  \defaultfontfeatures[\rmfamily]{Ligatures=TeX,Scale=1}
\fi
\usepackage{lmodern}
\ifPDFTeX\else
  % xetex/luatex font selection
\fi
% Use upquote if available, for straight quotes in verbatim environments
\IfFileExists{upquote.sty}{\usepackage{upquote}}{}
\IfFileExists{microtype.sty}{% use microtype if available
  \usepackage[]{microtype}
  \UseMicrotypeSet[protrusion]{basicmath} % disable protrusion for tt fonts
}{}
\makeatletter
\@ifundefined{KOMAClassName}{% if non-KOMA class
  \IfFileExists{parskip.sty}{%
    \usepackage{parskip}
  }{% else
    \setlength{\parindent}{0pt}
    \setlength{\parskip}{6pt plus 2pt minus 1pt}}
}{% if KOMA class
  \KOMAoptions{parskip=half}}
\makeatother
\usepackage{xcolor}
\usepackage[margin=1in]{geometry}
\usepackage{graphicx}
\makeatletter
\def\maxwidth{\ifdim\Gin@nat@width>\linewidth\linewidth\else\Gin@nat@width\fi}
\def\maxheight{\ifdim\Gin@nat@height>\textheight\textheight\else\Gin@nat@height\fi}
\makeatother
% Scale images if necessary, so that they will not overflow the page
% margins by default, and it is still possible to overwrite the defaults
% using explicit options in \includegraphics[width, height, ...]{}
\setkeys{Gin}{width=\maxwidth,height=\maxheight,keepaspectratio}
% Set default figure placement to htbp
\makeatletter
\def\fps@figure{htbp}
\makeatother
\setlength{\emergencystretch}{3em} % prevent overfull lines
\providecommand{\tightlist}{%
  \setlength{\itemsep}{0pt}\setlength{\parskip}{0pt}}
\setcounter{secnumdepth}{-\maxdimen} % remove section numbering
\ifLuaTeX
  \usepackage{selnolig}  % disable illegal ligatures
\fi
\usepackage{bookmark}
\IfFileExists{xurl.sty}{\usepackage{xurl}}{} % add URL line breaks if available
\urlstyle{same}
\hypersetup{
  pdftitle={CA2 Growth of Digital Games},
  hidelinks,
  pdfcreator={LaTeX via pandoc}}

\title{CA2 Growth of Digital Games}
\author{}
\date{\vspace{-2.5em}}

\begin{document}
\maketitle

\section{Introduction}\label{introduction}

The gaming industry has experienced a massive growth over the past few
decades, with the rise of Steam playing a pivitol role in shaping how
games are published and distributed. Steam hosts a large library of
games spanning over 90,000 with many genres, developers and price
points. With this vast choice for players and developers understanding
the trends and patterns in the data can be beneficial for both parties.
This report leverages the Steam Games Dataset 2025 which includes all
key information such as release dates, genre tags, developer and
publisher information, and user reviews. The dataset is used to explore
the growth of digital games on Steam and to identify the key trends and
patterns in the data.In this report I will be exploring these trends
using visualisation techniques in R, examples of things I will explore
are how pricing can effect user reviews and how certain genres have
grown and become more dominant over time. The findings will show a
broader understanding of how game developers and publishers strategise
their releases and how the pricing model may impact player perception
and how the genre lines up with current trends. This report will provide
meaningful insights for game developers and publishers to make informed
decisions on their game releases and marketing strategies.

\section{Research Questions /
Rational}\label{research-questions-rational}

\subsection{1. How does the frequeny / number of DLC's for a game effect
user
reviews?}\label{how-does-the-frequeny-number-of-dlcs-for-a-game-effect-user-reviews}

Some developers focus on keeping their games running for years by adding
gameplay mechanics with regular updates and DLC I want to investigate if
this has a positive or negative effect on user reviews. I aim to see if
the number of DLC's released for a game has an impact on the user
reviews and how this has changed over time. I think this is interesting
as some players may feel that the game is being milked for money and
that the DLC's should have been included in the base game. By analysing
this data I can see how the number of DLC's has impacted the user
reviews.

\subsection{2. How does the pricing of games affect user
reviews?}\label{how-does-the-pricing-of-games-affect-user-reviews}

The price of games has always been a crucial factor in players decision
making when buying a game but in this research question I want to figure
out how the pricing of games affects user reviews. I aim to see if there
is a correlation between the price of games and the user reviews and how
this has changed over time. I think this is interesting as higher priced
games tend to be reviewed more harshly by players due to the higher
expectations they have for the game. Players could also be more
forgiving of lower priced games as they are more likely to take a chance
on a cheaper game. By analysing this data I can see how the pricing of
games has impacted the user reviews.

\subsection{3. What genres have had trending growth over the
years?}\label{what-genres-have-had-trending-growth-over-the-years}

The third research question aims to understand what genres have had
trending growth over the years. I aim to see if certain genres have
become more popular over time and how this has impacted the sales of
games in those genres. This is important as it can help developers and
publishers understand what genres are popular and how they can
capitalise on this trend. I want to analyse the number of releases per
genre and see how this has changed over time by picking a few popular
genres and seeing how they have grown over the years compared to other
genres.

\subsection{4. How has the AAA landscape changed over the
years?}\label{how-has-the-aaa-landscape-changed-over-the-years}

The fourth research question aims to understand how the AAA landscape
has changed over the years. I aim to see how the number of AAA games has
increased over time and how this has impacted the sales of games in this
category. I also want to examine the price increase of these games and
the user reviews since these price changes have come into effect.I want
to understand the shift in the AAA landscape and see how the evolution
of these studios and game quality with regular price increases has
changed the publics opinion of AAA games and studios.

\includegraphics{CA2_files/figure-latex/unnamed-chunk-5-1.pdf}

\section{How does the frequeny / number of DLC's for a game effect user
reviews?}\label{how-does-the-frequeny-number-of-dlcs-for-a-game-effect-user-reviews-1}

The graph shows a general disconnect from the number of DLC's compared
to the number of positive reviews. The graph does show us some valuable
points though, we can see that for games with a very high review
percentage there is a general look that we can see that these games tend
to recieve small numbers of DLC's which indicates to me that the
developers of these games continued support for these games after
release due to positive perception. We can also see some games with a
lot of DLC's that have a lower review percentage which indicates to me
that these games are being milked for money and that the developers are
not putting in the effort to make the game better. This is a very
interesting trend as it shows that the number of DLC's does not always
correlate with the positive reviews and that some games can be
successful with a small number of DLC's while others can be unsuccessful
with a large number of DLC's.

\includegraphics{CA2_files/figure-latex/unnamed-chunk-6-1.pdf}

\section{How does the pricing of games affect user
reviews?}\label{how-does-the-pricing-of-games-affect-user-reviews-1}

Taking a deeper dive into the meaning of this graph I can see that there
is a non-linear relationship between the price of games and the
percentage of positive reviews. The graph shows the very cheap games
have a slightly lower positive percentage than the games in the range of
about 20-30. But we also see as games get above this range players seem
to be more harsh in the reviews as I think that these players have a
higher expectation of the game based on the price that they paid. We can
also see that there is outliers among the data as there is low rated
games in in all ranges which shows that no matter the price games can be
rated low. But in general we have seen that the price of games does have
an impact on player perception and forgiveness for flaws within games.

\includegraphics{CA2_files/figure-latex/unnamed-chunk-8-1.pdf}

\includegraphics{CA2_files/figure-latex/unnamed-chunk-9-1.pdf}
\includegraphics{CA2_files/figure-latex/unnamed-chunk-9-2.pdf}
\includegraphics{CA2_files/figure-latex/unnamed-chunk-9-3.pdf}
\includegraphics{CA2_files/figure-latex/unnamed-chunk-9-4.pdf}
\includegraphics{CA2_files/figure-latex/unnamed-chunk-9-5.pdf}
\includegraphics{CA2_files/figure-latex/unnamed-chunk-9-6.pdf}
\includegraphics{CA2_files/figure-latex/unnamed-chunk-9-7.pdf}
\includegraphics{CA2_files/figure-latex/unnamed-chunk-9-8.pdf}
\includegraphics{CA2_files/figure-latex/unnamed-chunk-9-9.pdf}
\includegraphics{CA2_files/figure-latex/unnamed-chunk-9-10.pdf}
\includegraphics{CA2_files/figure-latex/unnamed-chunk-9-11.pdf}
\includegraphics{CA2_files/figure-latex/unnamed-chunk-9-12.pdf}
\includegraphics{CA2_files/figure-latex/unnamed-chunk-9-13.pdf}

\includegraphics{CA2_files/figure-latex/unnamed-chunk-10-1.pdf}

\section{What genres have had trending growth over the
years?}\label{what-genres-have-had-trending-growth-over-the-years-1}

These graphs can show us a few interesting things about the trends of
game genres over the years, In the first graph we see the percentage of
games released in the top 5 genres over the years. We can see the
dominance of action games in the early 2010's and how the genre has
dropped off slightly in recent years. We can also see the rise of casual
games which in turn shows the growth of the gaming industry in these
years as more and more casual gamers have started playing which has
shifted developers focus to jump into the casual markets as shown in the
graph. We can also see a large amount of Indie games released more often
by smaller developers trying to make a name for themselves which is
still a very steady trend in the industry. We can also see the rise of
Massively Multiplayer games in the last few years which is a very
interesting trend as it shows that with the increase in computational
power and internet speeds that these games are becoming more and more
popular.I feel these graphs show a very interesting oversight into the
world of game genres and the trends that have come and gone over the
years.

\includegraphics{CA2_files/figure-latex/unnamed-chunk-11-1.pdf}

\includegraphics{CA2_files/figure-latex/unnamed-chunk-12-1.pdf}

\includegraphics{CA2_files/figure-latex/unnamed-chunk-13-1.pdf}

\section{How has the AAA landscape changed over the
years?}\label{how-has-the-aaa-landscape-changed-over-the-years-1}

The AAA landscape has changed a lot over the years as we can see from
the graphs above. The first graph shows the number of AAA games released
per year and we can see that there is a steady increase in the number of
AAA games released over the years.This shows that more AAA game
developers are showing up or else due to the way I have decided to
catagorise AAA games that the increase in price in games in recent years
has meant smaller studios games have come up in price in line with older
AAA titles. We can see from the graph that the price of these games has
steadily risen in recent years with the release of the PS5 and Xbox
series X we saw a increase in the price of games to 70 euro which is a
large increase from the previous generation of consoles. This has led to
a lot of backlash from players as they feel that the games are not worth
the price that they are being sold for. This is a similar trend I saw
previously in the pricing of games and how it affects user reviews. We
can see that the price of AAA games has increased over the years and
this has led to a lot of backlash from players as they feel that the
games are not worth the price that they are being sold for. This is a
similar trend I saw previously in the pricing of games and how it
affects user reviews. We can also see that the genres of AAA games have
changed over the years with the rise of Massively Multiplayer games for
AAA studios as they use these games as potential cash cows for the
studio as they use a live-service type game model to keep the game alive
and keep players coming back to the game. This is interesting as it
shows that the motivation for these big companies is to make money and
increase profits for shareholders.

\section{Conclusion}\label{conclusion}

This report provided an in-depth analysis on various digital games
development and distribution factors through analysis of the Steam Games
dataset --- from the impact of downloadable content (DLC), through
pricing models, to genre trends and the AAA game landscape evolution.

The results showed that although an abundance of DLCs does not
automatically translate to a better reception from users, continued
post-launch support in moderation is generally tied to positively
reviewed games. When it comes to pricing, though, the relationship is
distinctly non-linear: games sold for a little bit more than the average
get better average review scores than games sold at the bottom or the
very top of the pricing spectrum, while pricier games are held to higher
standards and typically received more brutal reviews on average.

The evolution of genres over the years mirrored the broader changes in
the gaming landscape. I explored the constant presence of action and
indie titles, the proliferation of casual games that could be attributed
to new gamers and mobile gamers, as well as the rising popularity of
massively multiplayer genres --- all signaling the market and
technological capability changing in their makings.

Finally, the AAA space has changed a lot. Given the growing tension
between player expectations and studio strategies as games continue to
sell for higher prices, face more scrutiny, and pivot to evermore
live-service work and monetisation, it ought to be no surprise that the
data reveals a difference between expectations and reality. This
reinforces the need for developers --- and particularly those in the AAA
market --- to toe the line between money-grabbing practices and
value-oriented game design.

Overall this shows that with the help of data, developers, publishers
and even gamers can better understand what is really going on in our
industry, where they can fit themselves into it as we enter an exciting
new period.

\section{References}\label{references}

\begin{itemize}
\tightlist
\item
  Steam Games Dataset 2025:
  \href{https://www.kaggle.com/datasets/nikhilroxtomar/steam-games-dataset-2025}{Steam
  Games Dataset}
\end{itemize}

\end{document}
